% ---------------------------------------------------------------------
% HEADER
% Formålet med å legge header til et eget dokument er å garantere at
% oppsettet av dokumentene er likt for alle løsningsforslagene.
% I headeren skjer følgende:
% (1) Dokumentet blir startet
% (2) Pakker blir importert
% ---------------------------------------------------------------------
% ---------------------------------------------------------------------
% HEADER
% Formålet med header er å importere de samme pakkene i alle dokumentene.
% ---------------------------------------------------------------------

% Sett opp dokumentet. Her kan 'twoside' brukes for printing
\documentclass[12pt, a4paper]{article}

% Vi trenger utf-8 for å bruke norske bokstaver: Æ, Ø, Å
\usepackage[utf8]{inputenc}

% Vi setter babel til norsk, da får dokumentegenskaper norske titler
\usepackage[norsk]{babel}

% For å kunne bruke grafikk
\usepackage{graphicx}
\newcommand{\figwidth}{0.75}

% Matematikkpakker fra AMS - American Mathematical Society
\usepackage{amsmath, amsthm, amsfonts, amssymb, mathtools}

% For eventuelle linker, e.g. \href{URL}{text}
\usepackage{hyperref}

% For headers og footers med eventuell logo
\usepackage{fancyhdr}

% Sett marginer manuelt
\usepackage[top = 3cm, left = 3cm, right = 3cm, bottom = 3cm]{geometry}

% For enkle lister, nyttig for oppgave a), b), c), ...
\usepackage[sharp]{easylist}

% Dersom flere kolonner er ønskelig i deler av dokumentet
\usepackage{multicol}

% For luft mellom paragrafer
\usepackage{parskip}

% For logikk assosiert med logoer
\usepackage{ifthen}

% For å finne totalt antall sider
\usepackage{lastpage}

% Annet
\usepackage{enumitem}

% Polynomer, og polynomdivisjon
\usepackage{polynom}
\polyset{style=C, div=:}

% Likningssystemer
\usepackage{systeme}

% Kan brukes når noe stryker ut noe, f.eks 1/n * n, her kan man ta \frac{1}{\cancel{n}} * \cancel{n}
\usepackage{cancel}

% SI enheter
\usepackage{siunitx}



% ---------------------------------------------------------------------
% DOKUMENTVARIABLER
% ---------------------------------------------------------------------
\newcommand{\fagkode}{S1}
\newcommand{\semesteraar}{våren 2018}
\newcommand{\forfatter}{Tommy O.}
\newcommand{\dokumenttittel}{Løsningsforslag -- Eksamen \fagkode, \semesteraar}


% Set til 'true' og oppgi logo dersom du vil bruke en logo
\newboolean{bruklogo}
\setboolean{bruklogo}{false}
\newcommand{\logonavn}{}

% ---------------------------------------------------------------------
% SETUP
% Formålet med å legge setup til et eget dokument å garantere at headers,
% footers, og øverste del av dokumentet er likt for alle
% løsningsforslagene.
% ---------------------------------------------------------------------
% ---------------------------------------------------------------------
% HEADER
% Formålet med setup er at dokumentene ser rimelig like ut.
% ---------------------------------------------------------------------


% ---------------------------------------------------------------------
% Alternativ font. Kommentert ut fordi Computer Modern (default) er pen
%\usepackage{kmath,kerkis}
%\usepackage[T1]{fontenc}
% ---------------------------------------------------------------------


% ---------------------------------------------------------------------
% Sett opp headers og footers
\ifthenelse{\boolean{bruklogo}}{
% Dersom logo skal brukes, sett logoen oppe til høyre med bredde 4 cm
	\rhead{\includegraphics[width=3.5cm]{\logonavn}}
}{
% Dersom logo ikke skal brukes, sett tom header
	\rhead{}
} 
\rfoot{\thepage}
\cfoot{}
\lhead{}
\lfoot{{\scriptsize Forbedringsforslag? Bidra på \url{https://github.com/tommyod/matte_eksamener_VGS}.}}
\renewcommand{\headrulewidth}{0pt}
% ---------------------------------------------------------------------


% ---------------------------------------------------------------------
% To streker under svaret
\def\answer#1{\underline{\underline{#1}}}
% ---------------------------------------------------------------------


% ---------------------------------------------------------------------
% Start selve dokumentet
% ---------------------------------------------------------------------

\begin{document}
\pagestyle{fancy}
{\bfseries \Large \dokumenttittel} \\
{ \footnotesize Laget av \forfatter 
	\hfill Sist oppdatert: \today 
	\hfill Antall sider: \pageref*{LastPage}}
\hrule
\vspace{1em}
\begin{center}
\fbox{\fbox{\parbox{.875\textwidth}{
	Dette dokumentet er open-source;
	alle kan bidra til å gjøre det bedre.
	Dersom du finner skrivefeil, matematiske feil, eller ser at forklaringene kan være bedre: ikke nøl med å bidra. 
	Du kan finne siste versjon, og bidra, på GitHub, se:
	\url{https://github.com/tommyod/matte_eksamener_VGS}
}}}
\end{center}


% ---------------------------------------------------------------------
% DOKUMENTSTART - Skriv løsningsforslaget nedenfor
% ---------------------------------------------------------------------	
\section*{Del 1 - uten hjelpemidler}
\subsection*{Oppgave 1}
\begin{easylist}[enumerate]
\ListProperties(Style2*=,Numbers=a,Numbers1=l,FinalMark={)})
# Vi flytter over på én side av likningen, slik at vi får en andregradslikning som vi kan faktorisere med ABC-formelen (eller en annen metode).
\begin{align*}
	2x^2 - 5x + 1 &= x - 3 \\
	2x^2 - 6x + 4 &= 0 \\
	2 \left( x^2 - 3x + 2 \right) &= 0 \\
	2 (x-1)(x-2) &= 0 \\
	\answer{x = 1 \text{ eller } x = 2} &	
\end{align*}
# Her flytter vi over slik at vi får logaritmen på alene, og deretter tar vi 10 opphøyd i begge sider av likningen for å bli kvitt logaritmen, fordi $10^{\lg x} = x$.
\begin{align*}
	2 \lg \left(x + 7\right) &= 4 \\
	\lg \left(x + 7\right) &= 2 \\
	10^{\lg \left(x + 7\right)} &= 10^2 \\
	x + 7 &= 100 \\
	\answer{x = 93}
\end{align*}
# Vi samler sammen så mye som mulig med samme grunntall, også bruker vi at dersom $2^x = 2^y$, så må $x=y$.
\begin{align*}
	3 \cdot 2^{3x+ 2} &= 12 \cdot 2^6 \\
	3 \cdot 2^{3x+ 2} &= 3 \cdot 2 \cdot 2 \cdot 2^6 \\
	3 \cdot 2^{3x+ 2} &= 3 \cdot 2^8 \\
	 2^{3x+ 2} &= 2^8 \\
	  {3x+ 2} &= 8 \\
	  \answer{x = 3}
\end{align*}
\end{easylist}

\subsection*{Oppgave 2}
Vi skal løse likningssystemet
\begin{align*}
	(1) \quad & x^2 + 3y = 7 \\	
	(2) \quad & 3x - y = 1.
\end{align*}
Vi løser likning $(2)$ for $y$ og får $y = 3x - 1$.
Dette setter vi inn i likning $(1)$, som gir oss
\begin{align*}
	x^2 + 3 \left( 3x - 1\right) &= 7 \\
	x^2 + 9x - 3 &= 7 \\
	x^2 + 9x - 10 &= 0 \\
	(x + 10)(x - 1) &= 0 \\
	x = -10 \text{ eller } x = 1.
\end{align*}
Vi bruker disse to $x$-verdiene til å finne $y$-verdier, vet å sette inn i likningen $y = 3x - 1$. Vi får da
\begin{align*}
	x = -10 & \Rightarrow y = 3(-10) - 1 = -31 \\
	x = 1 \quad \ & \Rightarrow y = 3(1) - 1 = 2
\end{align*}
Løsningene er altså $(x, y) = (-10, -31)$ og $(x, y) = (1, 2)$.


TODO: Lag en figur som viser.

\subsection*{Oppgave 3}
\begin{easylist}[enumerate]
	\ListProperties(Style2*=,Numbers=a,Numbers1=l,FinalMark={)})
	# Her ganger vi bare ut og kansellerer
	\begin{gather*}
		\left( 2x - 3\right)^2 - 2x \left(2x - 6\right) \\
		4x^2 - 12x + 9 - \left[ 4x^2 - 12x\right] \\
		4x^2 - 12x + 9 -  4x^2 + 12x \\
	9 \\
	\end{gather*}
	# Her må vi huske logaritmesetningene, altså at $\lg (a^x) = x \lg (a)$ og at $\lg \left(a b\right) = \lg (a) + \lg (b)$.
	\begin{gather*}
		\lg (2a) + \lg (4a) + \lg (8a) - \lg (16a) \\
		\lg (2) + \lg (a) + \lg (4) + \lg (a) + \lg (8) +\lg (a) - \left[\lg (16) + \lg (a)\right] \\ 
		\lg (2) + \lg (a) + 2\lg (2) + \lg (a) + 3\lg (2) +\lg (a) - \left[4 \lg (2) + \lg (a)\right] \\
		\lg (2) + \lg (a) + 2\lg (2) + \lg (a) + 3\lg (2) +\lg (a) - 4 \lg (2) - \lg (a) \\
		\answer{3 \lg (a) + 2 \lg (2)}
	\end{gather*}
	# Vi må finne fellesnevner for å legge sammen brøkene. Fellesnevneren er $ab$, så vi ganger den første bøken med $b$ og den andre med $b$ både i teller og nevner.
	
	\begin{equation*}
		\frac{1}{a} + \frac{1}{b} - \frac{a - b}{ab} =
		\frac{b}{ab} + \frac{a}{ab} - \frac{a - b}{ab} =
		\frac{b + a - (a - b)}{ab} =
		\frac{2b}{ab} = \answer{\frac{2}{a}} 
	\end{equation*}
\end{easylist}

\subsection*{Oppgave 4}
Vi bruker ABC-formelen eller en annen metode for å faktorisere andregradspolynomet. Et andregradspolynom har maksimalt to nullpunkter, og når leddet $x^2$ er positivt vokser funksjonen når $x$ blir veldig stor eller veldig liten---derfor vet vi at funksjonen er størst i halen.

\begin{equation*}
	x^2 - 3x + 2 \geq 0 
	\quad \Rightarrow \quad ( x - 1)(x - 2) \geq 0
	\quad \Rightarrow \quad \answer{x \leq 1 \text{ eller } x \geq 2}.
\end{equation*}

\subsection*{Oppgave 5}
\begin{easylist}[enumerate]
	\ListProperties(Style2*=,Numbers=a,Numbers1=l,FinalMark={)})
	# I Pascals trekant er hvert tall summen av de to tallene ovenfor.
	De første åtte radene ser slik ut.
	\begin{gather*}
	1 \\
	1 \quad 1 \\
	1 \quad 2 \quad 1 \\
	\textbf{1} \quad 3 \quad 3 \quad 1 \\
	1 \quad 4 \quad 6 \quad \textbf{4} \quad 1  \\
	1 \quad 5 \quad 10 \quad 10 \quad 5 \quad 1  \\
	1 \quad 6 \quad 15 \quad 20 \quad 15 \quad 6 \quad 1 \\
	1 \quad 7 \quad 21 \quad \textbf{35} \quad 35 \quad 21 \quad 7 \quad 1 
	\end{gather*}
	De tallene vi trenger i neste deloppgave er market med fet skrift.
	
	# Det er 3 røde kuler og 4 blå. Sannsynligheten for å trekke 3 blå er antall mulige måter å trekke 3 blå og 0 røde, delt på antall mulige måter å trekke 3 kuler totalt.
	\begin{equation*}
		P(3\text{ blå}) = \frac{\text{antall gunstige}}{\text{antall mulige}}
		= \frac{\binom{4}{3} \cdot \binom{3}{0}}{\binom{7}{3}} = 
		\frac{4 \cdot 1}{35} = \answer{\frac{4}{35}}
	\end{equation*}
	# Sannsynligheten for at det er minst én blå og minst én rød kan uttrykkes på følgende måte:
	\begin{align*}
		P(\text{minst én rød} \cap \text{minst én blå}) = 1 -
		& (P(\text{ingen røde} \cap \text{ingen blå}) + \\
		& P(\text{minst én rød} \cap \text{ingen blå}) + \\
		& P(\text{ingen røde} \cap \text{minst én blå}))
	\end{align*}
	TODO: Lag en figur som viser dette.
	Vi ser at
	\begin{align*}
		P(\text{ingen røde} \cap \text{ingen blå}) &= 0 \\
		P(\text{minst én rød} \cap \text{ingen blå}) &= \frac{1}{35} \\
		P(\text{ingen røde} \cap \text{minst én blå}) &= \frac{4}{35}
	\end{align*}
	og da blir 
	\begin{equation*}
		P(\text{minst én rød} \cap \text{minst én blå}) = 1 -
		\left( 0 + 
		\frac{1}{35} + 
		\frac{4}{35}\right) = \frac{30}{35} = \answer{\frac{6}{7}}
	\end{equation*}
\end{easylist}

\subsection*{Oppgave 6}
TODO: Lag figur som viser dette.

\subsection*{Oppgave 7}
\begin{easylist}[enumerate]
	\ListProperties(Style2*=,Numbers=a,Numbers1=l,FinalMark={)})
	# sdf
	# sdf
\end{easylist}

\subsection*{Oppgave 8}
\begin{easylist}[enumerate]
	\ListProperties(Style2*=,Numbers=a,Numbers1=l,FinalMark={)})
	# sdf
	# sdf
	# sdf
	# sdf
\end{easylist}

\subsection*{Oppgave 9}
\begin{easylist}[enumerate]
	\ListProperties(Style2*=,Numbers=a,Numbers1=l,FinalMark={)})
	# sdf
	# sdf
\end{easylist}





\section*{Del 2 - med hjelpemidler}


\subsection*{Oppgave 1}
asdf

\subsection*{Oppgave 2}
\begin{easylist}[enumerate]
	\ListProperties(Style2*=,Numbers=a,Numbers1=l,FinalMark={)})
	# sdf
	# sdf
	# sdf
\end{easylist}


\subsection*{Oppgave 3}
\begin{easylist}[enumerate]
	\ListProperties(Style2*=,Numbers=a,Numbers1=l,FinalMark={)})
	# sdf
	# sdf
	# sdf
	# sdf
\end{easylist}

\subsection*{Oppgave 4}
\begin{easylist}[enumerate]
	\ListProperties(Style2*=,Numbers=a,Numbers1=l,FinalMark={)})
	# sdf
	# sdf
	# sdf
	# sdf
\end{easylist}

\end{document}


