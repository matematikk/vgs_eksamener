% ---------------------------------------------------------------------
% HEADER
% Formålet med å legge header til et eget dokument er å garantere at
% oppsettet av dokumentene er likt for alle løsningsforslagene.
% I headeren skjer følgende:
% (1) Dokumentet blir startet
% (2) Pakker blir importert
% ---------------------------------------------------------------------
% ---------------------------------------------------------------------
% HEADER
% Formålet med header er å importere de samme pakkene i alle dokumentene.
% ---------------------------------------------------------------------

% Sett opp dokumentet. Her kan 'twoside' brukes for printing
\documentclass[12pt, a4paper]{article}

% Vi trenger utf-8 for å bruke norske bokstaver: Æ, Ø, Å
\usepackage[utf8]{inputenc}

% Vi setter babel til norsk, da får dokumentegenskaper norske titler
\usepackage[norsk]{babel}

% For å kunne bruke grafikk
\usepackage{graphicx}
\newcommand{\figwidth}{0.75}

% Matematikkpakker fra AMS - American Mathematical Society
\usepackage{amsmath, amsthm, amsfonts, amssymb, mathtools}

% For eventuelle linker, e.g. \href{URL}{text}
\usepackage{hyperref}

% For headers og footers med eventuell logo
\usepackage{fancyhdr}

% Sett marginer manuelt
\usepackage[top = 3cm, left = 3cm, right = 3cm, bottom = 3cm]{geometry}

% For enkle lister, nyttig for oppgave a), b), c), ...
\usepackage[sharp]{easylist}

% Dersom flere kolonner er ønskelig i deler av dokumentet
\usepackage{multicol}

% For luft mellom paragrafer
\usepackage{parskip}

% For logikk assosiert med logoer
\usepackage{ifthen}

% For å finne totalt antall sider
\usepackage{lastpage}

% Annet
\usepackage{enumitem}

% Polynomer, og polynomdivisjon
\usepackage{polynom}
\polyset{style=C, div=:}

% Likningssystemer
\usepackage{systeme}

% Kan brukes når noe stryker ut noe, f.eks 1/n * n, her kan man ta \frac{1}{\cancel{n}} * \cancel{n}
\usepackage{cancel}

% SI enheter
\usepackage{siunitx}



% ---------------------------------------------------------------------
% DOKUMENTVARIABLER
% ---------------------------------------------------------------------
\newcommand{\fagkode}{S2}
\newcommand{\semesteraar}{våren 2018}
\newcommand{\forfatter}{Tommy O.}
\newcommand{\dokumenttittel}{Løsningsforslag -- Eksamen \fagkode, \semesteraar}


% Set til 'true' og oppgi logo dersom du vil bruke en logo
\newboolean{bruklogo}
\setboolean{bruklogo}{true}
\newcommand{\logonavn}{figs/metis_akademiet_privatistskole_doclogo.png}

% ---------------------------------------------------------------------
% SETUP
% Formålet med å legge setup til et eget dokument å garantere at headers,
% footers, og øverste del av dokumentet er likt for alle
% løsningsforslagene.
% ---------------------------------------------------------------------
% ---------------------------------------------------------------------
% HEADER
% Formålet med setup er at dokumentene ser rimelig like ut.
% ---------------------------------------------------------------------


% ---------------------------------------------------------------------
% Alternativ font. Kommentert ut fordi Computer Modern (default) er pen
%\usepackage{kmath,kerkis}
%\usepackage[T1]{fontenc}
% ---------------------------------------------------------------------


% ---------------------------------------------------------------------
% Sett opp headers og footers
\ifthenelse{\boolean{bruklogo}}{
% Dersom logo skal brukes, sett logoen oppe til høyre med bredde 4 cm
	\rhead{\includegraphics[width=3.5cm]{\logonavn}}
}{
% Dersom logo ikke skal brukes, sett tom header
	\rhead{}
} 
\rfoot{\thepage}
\cfoot{}
\lhead{}
\lfoot{{\scriptsize Forbedringsforslag? Bidra på \url{https://github.com/tommyod/matte_eksamener_VGS}.}}
\renewcommand{\headrulewidth}{0pt}
% ---------------------------------------------------------------------


% ---------------------------------------------------------------------
% To streker under svaret
\def\answer#1{\underline{\underline{#1}}}
% ---------------------------------------------------------------------


% ---------------------------------------------------------------------
% Start selve dokumentet
% ---------------------------------------------------------------------

\begin{document}
\pagestyle{fancy}
{\bfseries \Large \dokumenttittel} \\
{ \footnotesize Laget av \forfatter 
	\hfill Sist oppdatert: \today 
	\hfill Antall sider: \pageref*{LastPage}}
\hrule
\vspace{1em}
\begin{center}
\fbox{\fbox{\parbox{.875\textwidth}{
	Dette dokumentet er open-source;
	alle kan bidra til å gjøre det bedre.
	Dersom du finner skrivefeil, matematiske feil, eller ser at forklaringene kan være bedre: ikke nøl med å bidra. 
	Du kan finne siste versjon, og bidra, på GitHub, se:
	\url{https://github.com/tommyod/matte_eksamener_VGS}
}}}
\end{center}


% ---------------------------------------------------------------------
% DOKUMENTSTART - Skriv løsningsforslaget nedenfor
% ---------------------------------------------------------------------	
\section*{Del 1 - uten hjelpemidler}
\subsection*{Oppgave 1}
\begin{easylist}[enumerate]
\ListProperties(Style2*=,Numbers=a,Numbers1=l,FinalMark={)})
# Vi skal derivere $f(x) = 2x^3 - 4x + 1$, og må bruke regelen $\left(x^n\right)' = nx^{n-1}$.
Vi får $f'(x) = 2(3)x^{3-1} - 4 x ^{1-1} + 0 = \answer{6x^2 - 4}$ som svar.

# Vi skal derivere $g(x) = x / e^x$. Det er fullt mulig å bruke brøkregelen for derivasjon, men man kan også skrive om funksjonen til produktet $g(x) = x e^{-x}$ og bruke produktregelen $(uv)' = u'v + uv'$ slik som dette
\begin{align*}
	g'(x) =& \left(x \right)' e^{-x} + \left( e^{-x} \right)' x \\
	=& 1 e^{-x} + (-1)e^{-x}  x \\
	=& e^{-x} + -1e^{-x}  x \\
	=& \answer{e^{-x}(1 - x)  }
\end{align*}
# Vi skal derivere $h(x) = \ln \left( x^2 + 4x\right)$, og må bruke kjernereglen $h'(x) = h'(u) \times u'(x)$, der $u$ er en kjerne.
Vi velger $u =x^2 + 4x$, da er $h(u) = \ln(u)$ og $h'(u) = 1/u)$, slik at vi får
\begin{align*}
	g'(x) =& h'(u) \times u'(x) \\
	=& \left( \frac{1}{u}  \right) \times \left( 2x + 4 \right) \\
	=& \frac{2x + 4}{u} =  \answer{\frac{2x + 4}{x^2 + 4x}}
\end{align*}
\end{easylist}

\subsection*{Oppgave 2}
Vi skal løse likningssystemet nedenfor, og vi kan bruke to forskjellige metoder: innsetningsmetoden eller addisjonsmetoden.
Vi velger addisjonsmetoden.
\begin{align*}
	5 x + y + 2 z &= 0 \quad (\text{A})\\
	2 x + 3 y + z &= 3 \quad (\text{B})\\
	3 x + 2 y - z &= -3 \quad (\text{C})
\end{align*}
For å kvitte oss med variabelen $z$ regner vi ut to nye likninger $(\text{D}) = (\text{A}) - 2 \times (\text{B})$ og $(\text{E}) = (\text{A}) + 2 \times (\text{C})$ som følger.
\begin{align*}
	x - 5y  &= -6 \quad (\text{D})\\
	11 x + 5 y  &= -6 \quad (\text{E})
\end{align*}
Ved å legge disse sammen kvitter vi oss med $y$, får likningen $12x = -12$ og ser at $\answer{x = -1}$.
For å løse for $y$ setter vi $x=-1$ inn i likning $(\text{D})$ eller $(\text{E})$ og ser at $\answer{y = 1}$.
Nå vet vi $x$ og $y$, og kan sette dette inn i $(\text{A})$, $(\text{B})$ eller $(\text{C})$ og finne ut at $\answer{z = 2}$.
På eksamen bør du sette en prøve å svaret---det går fort og du vet umiddelbart om du har regnet riktig.

\subsection*{Oppgave 3}
\begin{easylist}[enumerate]
	\ListProperties(Style2*=,Numbers=a,Numbers1=l,FinalMark={)})
	# asdf
	# asdf
\end{easylist}

\subsection*{Oppgave 4}
\begin{easylist}[enumerate]
	\ListProperties(Style2*=,Numbers=a,Numbers1=l,FinalMark={)})
	# asdf
	# asdf
\end{easylist}

\subsection*{Oppgave 5}
\begin{easylist}[enumerate]
	\ListProperties(Style2*=,Numbers=a,Numbers1=l,FinalMark={)})
	# asdf
	# asdf
\end{easylist}

\subsection*{Oppgave 6}
\begin{easylist}[enumerate]
	\ListProperties(Style2*=,Numbers=a,Numbers1=l,FinalMark={)})
	# asdf
	# asdf
	# asdf
	# asdf
\end{easylist}

\subsection*{Oppgave 7}
\begin{easylist}[enumerate]
	\ListProperties(Style2*=,Numbers=a,Numbers1=l,FinalMark={)})
	# asdf
	# asdf
\end{easylist}

\subsection*{Oppgave 8}
\begin{easylist}[enumerate]
	\ListProperties(Style2*=,Numbers=a,Numbers1=l,FinalMark={)})
	# asdf
	# asdf
\end{easylist}

\subsection*{Oppgave 9}
sdf







\section*{Del 2 - med hjelpemidler}

\subsection*{Oppgave 1}
\begin{easylist}[enumerate]
	\ListProperties(Style2*=,Numbers=a,Numbers1=l,FinalMark={)})
	# asdf
	# asdf
	# asdf
	# asdf
\end{easylist}

\subsection*{Oppgave 2}
\begin{easylist}[enumerate]
	\ListProperties(Style2*=,Numbers=a,Numbers1=l,FinalMark={)})
	# asdf
	# asdf
	# asdf
	# asdf
\end{easylist}

\subsection*{Oppgave 3}
\begin{easylist}[enumerate]
	\ListProperties(Style2*=,Numbers=a,Numbers1=l,FinalMark={)})
	# asdf
	# asdf
	# asdf
	# asdf
\end{easylist}




\end{document}


