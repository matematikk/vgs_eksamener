% ---------------------------------------------------------------------
% HEADER
% Formålet med å legge header til et eget dokument er å garantere at
% oppsettet av dokumentene er likt for alle løsningsforslagene.
% I headeren skjer følgende:
% (1) Dokumentet blir startet
% (2) Pakker blir importert
% ---------------------------------------------------------------------
% ---------------------------------------------------------------------
% HEADER
% Formålet med header er å importere de samme pakkene i alle dokumentene.
% ---------------------------------------------------------------------

% Sett opp dokumentet. Her kan 'twoside' brukes for printing
\documentclass[12pt, a4paper]{article}

% Vi trenger utf-8 for å bruke norske bokstaver: Æ, Ø, Å
\usepackage[utf8]{inputenc}

% Vi setter babel til norsk, da får dokumentegenskaper norske titler
\usepackage[norsk]{babel}

% For å kunne bruke grafikk
\usepackage{graphicx}
\newcommand{\figwidth}{0.75}

% Matematikkpakker fra AMS - American Mathematical Society
\usepackage{amsmath, amsthm, amsfonts, amssymb, mathtools}

% For eventuelle linker, e.g. \href{URL}{text}
\usepackage{hyperref}

% For headers og footers med eventuell logo
\usepackage{fancyhdr}

% Sett marginer manuelt
\usepackage[top = 3cm, left = 3cm, right = 3cm, bottom = 3cm]{geometry}

% For enkle lister, nyttig for oppgave a), b), c), ...
\usepackage[sharp]{easylist}

% Dersom flere kolonner er ønskelig i deler av dokumentet
\usepackage{multicol}

% For luft mellom paragrafer
\usepackage{parskip}

% For logikk assosiert med logoer
\usepackage{ifthen}

% For å finne totalt antall sider
\usepackage{lastpage}

% Annet
\usepackage{enumitem}

% Polynomer, og polynomdivisjon
\usepackage{polynom}
\polyset{style=C, div=:}

% Likningssystemer
\usepackage{systeme}

% Kan brukes når noe stryker ut noe, f.eks 1/n * n, her kan man ta \frac{1}{\cancel{n}} * \cancel{n}
\usepackage{cancel}

% SI enheter
\usepackage{siunitx}

\usepackage{tikz}

% ---------------------------------------------------------------------
% DOKUMENTVARIABLER
% ---------------------------------------------------------------------
\newcommand{\fagkode}{R1}
\newcommand{\semesteraar}{høsten 2018}
\newcommand{\forfatter}{Sindre S.H. og Tommy O.}
\newcommand{\dokumenttittel}{Løsningsforslag -- Eksamen \fagkode, \semesteraar}

% Set til 'true' og oppgi logo dersom du vil bruke en logo
\newboolean{bruklogo}
\setboolean{bruklogo}{false}
\newcommand{\logonavn}{}
\newcommand{\alg}[1]{
\begin{align*}
#1
\end{align*}
}


% ---------------------------------------------------------------------
% SETUP
% Formålet med å legge setup til et eget dokument å garantere at headers,
% footers, og øverste del av dokumentet er likt for alle
% løsningsforslagene.
% ---------------------------------------------------------------------
% ---------------------------------------------------------------------
% HEADER
% Formålet med setup er at dokumentene ser rimelig like ut.
% ---------------------------------------------------------------------


% ---------------------------------------------------------------------
% Alternativ font. Kommentert ut fordi Computer Modern (default) er pen
%\usepackage{kmath,kerkis}
%\usepackage[T1]{fontenc}
% ---------------------------------------------------------------------


% ---------------------------------------------------------------------
% Sett opp headers og footers
\ifthenelse{\boolean{bruklogo}}{
% Dersom logo skal brukes, sett logoen oppe til høyre med bredde 4 cm
	\rhead{\includegraphics[width=3.5cm]{\logonavn}}
}{
% Dersom logo ikke skal brukes, sett tom header
	\rhead{}
} 
\rfoot{\thepage}
\cfoot{}
\lhead{}
\lfoot{{\scriptsize Forbedringsforslag? Bidra på \url{https://github.com/tommyod/matte_eksamener_VGS}.}}
\renewcommand{\headrulewidth}{0pt}
% ---------------------------------------------------------------------


% ---------------------------------------------------------------------
% To streker under svaret
\def\answer#1{\underline{\underline{#1}}}
% ---------------------------------------------------------------------


% ---------------------------------------------------------------------
% Start selve dokumentet
% ---------------------------------------------------------------------

\begin{document}
\pagestyle{fancy}
{\bfseries \Large \dokumenttittel} \\
{ \footnotesize Laget av \forfatter 
	\hfill Sist oppdatert: \today 
	\hfill Antall sider: \pageref*{LastPage}}
\hrule
\vspace{1em}
\begin{center}
\fbox{\fbox{\parbox{.875\textwidth}{
	Dette dokumentet er open-source;
	alle kan bidra til å gjøre det bedre.
	Dersom du finner skrivefeil, matematiske feil, eller ser at forklaringene kan være bedre: ikke nøl med å bidra. 
	Du kan finne siste versjon, og bidra, på GitHub, se:
	\url{https://github.com/tommyod/matte_eksamener_VGS}
}}}
\end{center}


% ---------------------------------------------------------------------
% DOKUMENTSTART - Skriv løsningsforslaget nedenfor
% ---------------------------------------------------------------------	
\section*{Del 1 - uten hjelpemidler}
\subsection*{Oppgave 1}
\begin{easylist}[enumerate]
	\ListProperties(Style2*=,Numbers=a,Numbers1=l,FinalMark={)})
	# \begin{align*}
		f'(x) &= 2x + 2 + e^x
	\end{align*}
	# Vi bruker produktregelen ved derivasjon og får at:
	 \alg{
		g'(x) &= \left(x^2\right)'\ln x + x^2(\ln x)' 	\\
		&= 2x \ln x + x^2 \frac{1}{x} \\
		&= 2x \ln x + x \\
		&= x(2\ln x + 1)
}
	# \textit{Alternativ 1: } \\
	Vi omskriver $ h(x) $ ved hjelp av potensregler:
	\alg{
	h(x) &= (x-1)e^{-(2x+1)}
}
Videre merker vi oss at av kjerneregelen er:
\[ \left(e^{-(2x+1)}\right)'=-2e^{-(2x+1)}\]
Av produktregelen får vi da at:
\alg{
h'(x) &= (x-1)'e^{-(2x+1)}+(x-1)\left(e^{-(2x+1)}\right)'  \\
&= 1\cdot e^{-(2x+1)} - 2(x-1)e^{-(2x+1)} \\
&= (1-2(x-1))e^{-(2x+1)} \\
&= (3-2x)e^{-(2x+1)}
}
\textit{Alternativ 2}\\
Vi har at:
\alg{
(x-1)' &= 1\\
\left(e^{2x+1}\right)' &= 2e^{2x+1} \tag{kjerneregelen}
}
Av kvotientregelen ved derivasjon har vi da at:
\alg{
h'(x) &= \frac{(x-1)'e^{2x+1}-(x-1)\left(e^{2x+1}\right)'}{\left(e^{2x+1}\right)^2} \\
&= \frac{1\cdot e^{2x+1} - (x-1)2e^{2x+1}}{e^{2(2x+1)}
} \\
&= \frac{e^{2x+1}(1-(x-2))}{e^{2(2x+1)}}\\
&= \frac{3-2x}{e^{2x+1}}
} 

\end{easylist}


\subsection*{Oppgave 2}
\begin{easylist}[enumerate]
	\ListProperties(Style2*=,Numbers=a,Numbers1=l,FinalMark={)})
	# Vi merker oss at:
	\[ e^{2x}+7e^x -8 = \left(e^ x\right)^2 + 7e^x - 8 =0\]
	Dette betyr at vi kan løse en andregradsligning mhp. $ e^x $. Vi viser her to måter å løse ligningen på:\\[11pt]
	\textit{Alternativ 1} \\
	Ved abc-formelen har vi at:
	\alg{
	e^x &= \frac{-7\pm \sqrt{7^2-4\cdot1\cdot(-8)}}{2\cdot1}	 \\
	&= \frac{-7\pm \sqrt{81}}{2} \\
	&= \frac{-7\pm 9}{2}\\
}
\textit{Alternativ 2}\\
Siden $ 8(-1)=-8 $ og $ 8-1=7 $, har vi at:
\[ (e^x +8)(e^x-1)=0 \]
Både \textit{Alternativ 1} og \textit{Alternativ 2} gir at:
\[ e^x = 1\quad\vee\quad e^x = -8 \]
Siden $ e^x > 0 $, står vi bare igjen med løsningen $ e^x=1 $:
\alg{
\ln e^x &= \ln 1 \\
 x &= 0 
}
# Ved hjelp av logaritmeregler omskriver vi ligningen til:
\alg{\ln \left(\frac{x^2-5x-1}{3-2x}\right)=0
}
Videre får vi da at:
\alg{
e^{\ln \left(\frac{x^2-5x-1}{3-2x}\right)}=e^0 \\
\frac{x^2-5x-1}{3-2x} &= 1 \\
x^2-5x-1 &= 3-2x \\
x^2-3x-4 &= 
}
Fordi $ (-4)1 =-4 $ og $ -4+1=-3 $ har vi at:
\[ x^2-3x-4 = (x-4)(x+1)=0 \]
Siden $ x=4 $ gir negativ ln-verdi i originaluttrykket til ligningen, står vi bare igjen med $ x=-1 $ som gyldig løsning.
\end{easylist}

\subsection*{Oppgave 3}
\begin{easylist}[enumerate]
	\ListProperties(Style2*=,Numbers=a,Numbers1=l,FinalMark={)})
	# \alg{
		2\vec{b}-3\vec{a} &= 2[-5, 3] - 3[2, 3]	\\
		&= [-10, 6]-[6, 9] \\
		&= [-16, -3]
} 
	# En stump vinkel har negativ cosinusverdi, en rett vinkel har cosinusverdien 0 og en spiss vinkel har positiv cosinusverdi. Vi har at: \alg{
|\vec{a}| &= \sqrt{2^2 + 3^2} \\
&= \sqrt{13}	
}
Sidan $ \sqrt{13}<\sqrt{16}=4 $ er $ |\vec{a}| $ mindre enn 4.
	# Vi har at:
	\alg{\angle(\vec{a}, \vec{b}) &= \frac{\vec{a}\cdot\vec{b}}{|\vec{a}||\vec{b}|}
}
og videre at:
\[ |\vec{a}|, \vec{b} > 0\quad,\quad\vec{a}\cdot\vec{b}=2(-5)+3\cdot3=-1\quad \Rightarrow\quad \angle(\vec{a}, \vec{b}) <0 \]
Ergo er $ \angle(\vec{a}, \vec{b}) $ en stump vinkel.
\end{easylist}


\subsection*{Oppgave 4}
\begin{easylist}[enumerate]
	\ListProperties(Style2*=,Numbers=a,Numbers1=l,FinalMark={)})
	# \polylongdiv{x^3+6x^2-x-30}{x-2}
	# Siden $ 3\cdot5 = 15 $ og $ 3+5=8 $, har vi at:
	\[ x^2+8x+15 = (x+3)(x+5) \]
	Dermed er:
	\[x^3+6x^2-x-30=  (x+3)(x+5)(x-2) \]
	# At $ -2 f(x)\geq 0 $ er ekvivalent med at $ f(x)\leq 0 $. Vi setter opp et fortegnsskjema for å løse denne ulikheten:
		\begin{figure}[h]
		\centering
		\begin{tikzpicture}[scale=2]	
	% Start
		\draw[color=black] (0,0.25) -- (0,-1.75);
		
		\node[anchor=east] at (0,0) { $(x+3)$};  
	    \node[anchor=east] at (0,-0.5) { $(x+5)$};  		
	  	\node[anchor=east] at (0,-1) {$(x-2)$};       	
	  	\node[anchor=east] at (0,-1.5) {$f(x)$};   
	
	% Til -5
		\node[anchor=south] at (1,0.25) { $-5$};    		
		\draw[color=black] (1,0.25) -- (1,-1.75);	
		
  		\draw[dashed,color=black] (0,0) -- (1,0);
  		\draw[dashed,color=black] (0,-0.5) -- (1,-0.5);
  		\filldraw (1,-0.5) circle[radius=1pt] ;  
  		\draw[dashed,color=black] (0,-1) -- (1,-1);
  		\draw[dashed,color=black] (0,-1.5) -- (1,-1.5);
  		\filldraw (1,-1.5) circle[radius=1pt] ;  

	% Til -5
		\node[anchor=south] at (2,0.25) { $-3$};    		
		\draw[color=black] (2,0.25) -- (2,-1.75);	
		
		\draw[dashed,color=black] (1,0) -- (2,0);
		\draw[color=black] (1,-0.5) -- (2,-0.5);
		\filldraw (2,0) circle[radius=1pt] ;  
		\draw[dashed,color=black] (1,-1) -- (2,-1);	
		\draw[color=black] (1,-1.5) -- (2,-1.5);
		\filldraw (2,-1.5) circle[radius=1pt] ;  		
	
	% Til 2
		\node[anchor=south] at (3,0.25) { $2$};    		
		\draw[color=black] (3,0.25) -- (3,-1.75);	
		
		\draw[color=black] (2,0) -- (3,0);
		\draw[color=black] (2,-0.5) -- (3,-0.5);
		\filldraw (3,-1) circle[radius=1pt] ;  
		\draw[dashed,color=black] (2,-1) -- (3,-1);			
		\draw[dashed, color=black] (2,-1.5) -- (3,-1.5);
		\filldraw (3,-1.5) circle[radius=1pt] ;  			

	% Slutt	
		\draw[color=black] (4,0.25) -- (4,-1.75);	
		\draw[color=black] (3,0) -- (4,0);
		\draw[color=black] (3,-0.5) -- (4,-0.5);
		\draw[color=black] (3,-1) -- (4,-1);		
		\draw[color=black] (3,-1.5) -- (4,-1.5);
		\filldraw (3,-1.5) circle[radius=1pt] ;  			
		\end{tikzpicture}
	\end{figure}
Av figuren under ser vi at $ f(x)\leq 0 $ når:
\[ x\leq -5 \quad \vee \quad -3\leq x \leq 2 \]
\end{easylist}

\newpage
\subsection*{Oppgave 5}
\begin{easylist}[enumerate]
Vi definerer følgende:
\alg{
	A &= \text{Mann}\quad&\bar A &= \text{Kvinne}	\\
	B &= \text{Kjøper edelgran}\quad &\bar B &= \text{Kjøper Gran}
}
Da har vi at: 
\alg{
	P(A) &= 70\% \quad&\bar P(\bar A) &= 30\%	\\
	P(B|A) &=60\%\quad & P( B|\bar A) &= 40\% 
}	
	\ListProperties(Style2*=,Numbers=a,Numbers1=l,FinalMark={)})
	# 
Av formelen for total sannsynlighet har vi at:
\alg{
P(B) &= P(A)P(B|A)+P(\bar A)P(B|\bar A) \\
&= 0.7\cdot0.6 + 0.3\cdot0.4\\
&= 0.54
}
Det er altså 54\% sannsynlighet for at første solgte tre er en edelgran.
	# Spørsmålet i oppgaven er det samme som å spørre hva sannsynligheten er for at kjøperen av et tre er en kvinne, gitt at treet er en edelgran. Av Bayes' setning har vi at:
	\alg{
	P(\bar A|B)	&= \frac{P(B|\bar A)\cdot P(\bar A)}{P(B)} \\
	&= \frac{0.4\cdot 0.3}{0.54} \\[5pt]
	&= \frac{12}{54}\\[5pt]
	&= \frac{2}{9}
}
Sannsynligheten er altså $ \frac{2}{9} $ for at lotterivinneren er en kvinne.
\end{easylist}


\subsection*{Oppgave 6}
Skal $ f(x) $ være kontinuerlig, må:
\[ \lim\limits_{x^\pm\to a} f(x) =f(a)\]
I vårt tilfelle er det tilstrekkelig å sørge for at $ f(a) $ får samme verdi ved begge tilfeller av funksjonsuttrykkene for $ f(x) $:
\alg{
	2a^2 - 3a - 2 &= a^2 +a + 3\\
	a^2 -4a-5 &= 0
}
Siden $ (-5)1=-5 $ og $-5+1=-4 $, har vi at:
\[ (a-5)(a+1)=0 \]
Dette gir at $ f(x) $ er kontinuerlig når:
\[ a = 5 \quad\vee\quad a=-1 \]
\subsection*{Oppgave 7}
\begin{easylist}[enumerate]
	\ListProperties(Style2*=,Numbers=a,Numbers1=l,FinalMark={)})
	# Vi bruker kjerneregelen og får at:
	\alg{
\left(\ln(x^2+3)\right)'&= 2x\frac{1}{x^2+3}	\\
&= \frac{2x}{x^2+3}
}
Videre har vi da at:
\alg{
g'(x) &= 1-2\cdot\frac{2x}{x^2+3} \\
&= \frac{x^2+3-4x}{x^2+3} \\
&= \frac{x^2-4x+3}{x^2+3}
}
Som var det vi skulle vise.
	# For $ x $-verdien til toppunkt og bunnpunkt er $ g'(x)=0 $. Fordi $ (-1)(-3)=3 $ og $ -1+(-3)=-4 $ kan vi omskrive $ g'(x) $, og får da at:
	\alg{
	g'(x) &= 0 \\
	\frac{(x-1)(x+3)}{x^2+3}&= 0 \\
	(x-1)(x-3) &= 0
}
Ved å sette opp et fortegnsskjema (ikke vist her, se opg. 4c) finner vi at $ x=1 $ er et maksimalpunkt, mens $ x=3 $ er et minimumspunkt.
	# I et vendepunkt må $ g''(x)=0 $. $ g''(x) $ finner vi ved å derivere $ g'(x) $ ved hjelp av kvotientregelen:
	\alg{
g'(x)	&= \frac{\left(x^2-4x+3\right)'\left(x^2+3\right)-\left(x^2-4x+3\right)\left(x^2+3\right)'}{\left(x^2+3\right)^2}\\
\left(x^2+3\right)^2g''(x)&= (2x-4)(x^2+3)-2x(x^2-4x+3)\\
&=2x^3+6x-4x^2-12-2x^3+8x^2-6x \\
&= 4x^2 - 12
}
Skal $ g''(x)=0 $ må vi kreve at høyresiden i ligningen over blir 0:
\alg{
4x^2 - 12 &= 0 \\
x^2 &= 3
}
Altså har $ g(x) $ infleksjonspunkter der hvor:
\[ x = \sqrt{3} \quad \vee\quad x=-\sqrt{3} \]

\end{easylist}


\subsection*{Oppgave 8}
\begin{easylist}[enumerate]
	\ListProperties(Style2*=,Numbers=a,Numbers1=l,FinalMark={)})
	# sdf
	# sdf
	# sdf
\end{easylist}







\clearpage 
\section*{Del 2 - med hjelpemidler}


\subsection*{Oppgave 1}
La $X$ være antall gule blomster.
Da er $X$ binomisk fordelt, fordi fargen til hver blomst er uavhengig av de andre, sannsynligheten for gul er alltid $p = 0.4$, og det er kun to utfall per blomst---enten gul eller rød.

\begin{easylist}[enumerate]
	\ListProperties(Style2*=,Numbers=a,Numbers1=l,FinalMark={)})
	# Vi lar $X$ være antall gule blomster, da er $X$ binomisk fordelt med $n=10$ og $p=0.4$. 
	Sannsynligheten for at $X=5$ blir da
	\begin{equation*}
	P(X = 5) = \binom{n}{k} p^k (1-p)^{n-k} = 
	\binom{10}{5} 0.4^5 0.6^{5} \approx 0.2007 = \answer{20.1 \, \%},
	\end{equation*}
	der vi henter svaret fra Geogebras sannsynlighetskalkulator i praksis.
	# Dette er $P(X > 5)$, og fra sannsynlighetskalkulatoren i Geogebra får vi
	\begin{equation*}
	P(X > 5) \approx 0.1662 = \answer{16.6 \, \%} .
	\end{equation*}
	# Her kan det være nyttig å snu litt på problemstillingen.
	Vi har 10 plasser $P_1, P_2, \dots, P_{10}$ totalt, og vi må trekke ut fire plasser til de gule blomstene.
	Da er plasseringen til de andre blomstene også bestemt.
	Rekkefølgen har ikke noe å si, fordi $\{ P_3, P_5, P_6, P_9 \}$ eksempelvis er det samme som $\{ P_6, P_5, P_3, P_9 \}$.
	Spørsmålet er med andre ord ``På hvor mange måter kan vi velge 4 plasser for de gule blomstene, nå rekkefølgen er uviktig?''.
	Dette er antall kombinasjoner, og svaret blir $\binom{10}{4} = \answer{210}$.
	I Geogebra skriver man \verb|nCr(10, 4)| i CAS.
	
	Legg også merke til at dette spørsmålet er det samme som å spørre ``På hvor mange måter kan vi velge 6 plasser for de røde blomstene, nå rekkefølgen er uviktig?'', og at svaret da blir $\binom{10}{6} = \answer{210}$.
\end{easylist}

\subsection*{Oppgave 2}
\begin{easylist}[enumerate]
	\ListProperties(Style2*=,Numbers=a,Numbers1=l,FinalMark={)})
	#  Siden $ CB|| AE $ og $ \angle BCD $ og $ \angle AED $ er på motsatt side av $ AE $, er $ \angle BCD = \angle AED $ 
	# \begin{itemize}
		\item Vi har forklart i oppgave a) at $  \angle BCD = \angle AED$
		\item Siden $ \angle ADE $ og $ \angle BDC $ er på motsatt side av $ AE $, og begge vinklene deler linja $ AB $, er $ \angle BCD = \angle AED $ 
		\item Fordi trekantene har to samsvarende vinkler, er $ \triangle DBC \sim \triangle AED $ (formlike).
	\end{itemize}
	# $ \triangle AEC $ består blant annet av vinklene $ ACD $ og $ AED. $ Siden $ \angle AED = \angle BCD = \alpha = ACD $, har $ \triangle AEC $ to like vinkler og er derfor likebeint.
	# Siden $ \triangle DBC \sim \triangle AED $ er forholdet mellom to samsvarande sider i trekantene likt. $ \angle AED$ og $ \angle BCD $ utspenner respektivt linjene $ AD $ og $ DB $, mens $ \angle ADE$ og $ \angle CDB $ utspenner respektivt linjene $ AE $ og $ a$. Da har vi at:
	\[ \frac{AD}{DB}= \frac{AE}{a}\]
	Siden $ \triangle ACE $ er likebeint er $ AE=b $:
	\alg{
		\frac{AD}{DB}= \frac{b}{a}
}
Som var det vi skulle vise.
	# Siden $ c=10 $, har vi at $ DB=10-AD $. Dermed kan vi bruke forholdet fra opg. d) til å løse en ligning mhp. $ AD $. Lignigen er løst i CAS:
	\begin{figure}[h]
		\centering
		\includegraphics[scale=0.6]{figs/D2opg2e}
		\caption{Ligning for $ AD $ løst i CAS}
	\end{figure}
\end{easylist}



\subsection*{Oppgave 3}
\begin{easylist}[enumerate]
	\ListProperties(Style2*=,Numbers=a,Numbers1=l,FinalMark={)})
	# Linjen $\ell$ er gitt av
	\begin{equation*}
	\ell(k) = A + \vec{AB} k = (3, 0) + \left[ (5, 5) - (3, 0) \right] k = (2k+3, 5k).
	\end{equation*}
	# Se figur \ref{fig:del2_oppg3} for en tegning av grafen, utført med kommandoen \\
	\verb|Kurve(<Uttrykk>, <Uttrykk>, <Parametervariabel>, <Start>, <Slutt>)|,\\
	samt linjen $\ell(k)$.
	# Linja fra $A$ til $B$ er gitt av $\vec{AB} = (2, 5)$.
	Tangenten er gitt av $T(t) = r'(t) = (1, 2t)$.
	Disse to vektorene er parallelle dersom der finnes en $z$ slik at
	\begin{equation*}
	\begin{pmatrix}
	2 \\
	5
	\end{pmatrix} z =
	\begin{pmatrix}
	1 \\
	2t
	\end{pmatrix}.
	\end{equation*}
	Dette er et likningssett med 2 ukjente og 2 likninger, og løsningen er $z = 1/2$ og $t = 5/4$.
	Setter vi $t = 5/4$ inn i $r(t)$ får vi punktet 
	\begin{equation*}
	\answer{\begin{pmatrix}
		9/4 \\
		57/6
		\end{pmatrix}},
	\end{equation*}
	og dette er punktet på $r(t)$ som er nærmest $\ell$.
\end{easylist}

\begin{figure}[ht!]
	\centering
%	\includegraphics[width=0.9\linewidth]{figs/del2_oppg3}
	\caption{Løsning på oppgave 3, del 2.}
	\label{fig:del2_oppg3}
\end{figure}

\subsection*{Oppgave 4}
\begin{easylist}[enumerate]
	\ListProperties(Style2*=,Numbers=a,Numbers1=l,FinalMark={)})
	# Se figur \ref{fig:del2_oppg4_a}.
	
	\begin{figure}[ht!]
		\centering
%		\includegraphics[width=0.9\linewidth]{figs/del2_oppg4_a}
		\caption{Løsning på oppgave 4a, del 2.}
		\label{fig:del2_oppg4_a}
	\end{figure}
	
	# Der bruker vi \verb|Mangekant(A, B, C) / Mangekant(A, B, D)| og får $2.5981 \approx \answer{2.6}$ som svar.
	
	# \begin{itemize}
		\item I celle 1 finner vi nullpunktene til $ g(x) $. Siden $ r>0 $, er $ x=-r $ nullpunktet lengst til venstre.
		\item I celle 5-7 finner og definerer vi punktet $ G $.
		\item I celle 8 finner vi ekstremalpunktene til $ g(x) $. Siden $ \left|\frac{\sqrt{3}}{3}r\right|<r$, ser vi av uttrykket til $ g(x) $ at $ x=-\frac{\sqrt{3}}{3}r $ er maksimusverdien til $ g(x) $. Denne kaller vi $ xm $ i celle 9.
		\item I celle 10 finner vi koordinatene til $ H $.
		\item Vi velger $ EF $ som grunnlinje for begge trekanter, som da får høydene $ t(-r) $ og $ g(xm) $. Siden grunnlinjen er den samme, er forholdet mellom arealene til trekantene det samme som forholdet mellom høydene. I celle 11 ser vi at dette forholdet er uavhengig av $ r $, som var det vi skulle vise.
	\end{itemize}\begin{figure}
		\centering
		\includegraphics[scale=0.7]{figs/d2opg4c}
	\end{figure}
\end{easylist}





\end{document}